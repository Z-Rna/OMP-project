\documentclass{article}
\usepackage[T1]{fontenc}
\usepackage{scrextend}

\begin{document}
    \title{
        Sprawozdanie z Projektu Pierwszego \\
        \small{Laboratorium Przetwarzania Równoległego}
    }
    \author{
        \textbf{Piotr Tylczyński}\\
        \texttt{L7 / 141331} \\
        \texttt{Środa, 11:45} \\
        \texttt{piotr.tylczynski@student.put.poznan.pl}
        \and
        \textbf{Zuzanna Rękawek}\\
        \texttt{L7 / 141304} \\
        \texttt{Środa, 11:45} \\
        \texttt{zuzanna.rekawek@student.put.poznan.pl}
        \date{}
        }
    \begin{titlepage}
        \maketitle
        \centering{
            Oddane: 29.04.2021 \\
            Deadline: 29.04.2021 \\
        }
        \hfill \break
        \centering{Wersja 1}
            
            
    \end{titlepage}
    
    \tableofcontents
    \pagebreak
    
    \section{Motywacja}
        Celem niniejszego projketu jest stworzenie efektywnego programu wyszukującego liczby pierwsze w zadanym przedziale. W tym celu wykorzystamy porogramowanie równoległe. Pozwoli to na efektywniejsze wykorzystanie zasobów komputerowych jakimi dysponujemy. W wyniku otrzymamy program mogący wykorzystywać do 100\% mocy obliczeniowej procesora komputera, na którym zostanie uruchomiony. Pozwoli to nam na znaczącą redukcję czasu wykonania programu względem standardowej wersji sekwencyjnej programu.
        
        W rozwiązaniu sotsujemy algorytm Sita Erastotenesa \emph{(SE)}, oraz pełnego przeglądu wszystkich możliwych dzielników \emph{(PPD)} danej liczby. Oba algorytmy mają olbzymi potencjał zrównoleglenia, jednak szczególną uwagę poświęcimy zagadnieniu zrównoleglania i badania jego efektów dla Sita Erastotenesa.
                 
    \section{Specyfikacja platformy uruchomieniowej}
        \begin{addmargin}{3em}
            \begin{description}
                \item[Procesor] Intel Core i5-9300H
                    \begin{description}
                        \item[Procesorów Fizycznych] 4
                        \item[Procesorów Logicznych]  8
                        \item[Pamięć Cache]  8 MB Intel® Smart Cache 
                    \end{description} 
                \item[System Operacyjny] Windows 10 Pro 20H2
                \item[IDE] Visual Studio 2019
                % TODO: check vtune version
                \item[Oprogramowanie Testujące] 5t4iori
            \end{description}
        \end{addmargin}

        
    \section{Zastosowane Algorytmy}
        \subsection{Opis teoretyczny}
            \subsubsection{Algorytm Sekwencyjny}
                
        \subsection{Realizacja praktyczna}
\end{document}