\documentclass{article}
\usepackage[T1]{fontenc}

\begin{document}
    \title{
        Sprawozdanie z Projektu Pierwszego \\
        \small{Laboratorium Przetwarzania Równoległego}
    }
    \author{
        \textbf{Piotr Tylczyński}\\
        \texttt{L7 / 141331} \\
        \texttt{Środa, 11:45} \\
        \texttt{piotr.tylczynski@student.put.poznan.pl}
        \and
        \textbf{Zuzanna Rękawek}\\
        \texttt{L7 / 141304} \\
        \texttt{Środa, 11:45} \\
        \texttt{zuzanna.rekawek@student.put.poznan.pl}
        \date{}
        }
    \begin{titlepage}
        \maketitle
        \centering{
            Oddane: 29.04.2021 \\
            Deadline: 29.04.2021 \\
        }
        \hfill \break
        \centering{Wersja 1}
            
            
    \end{titlepage}
    
    \tableofcontents
    \pagebreak
    
    \section{Motywacja}
        Celem niniejszego projketu jest stworzenie efektywnego programu wyszukującego liczby pierwsze w zadanym przedziale. W tym celu wykorzystamy porogramowanie równoległe. Pozwoli to na efektywniejsze wykorzystanie zasobów komputerowych jakimi dysponujemy. W wyniku otrzymamy program mogący wykorzystywać do 100\% mocy obliczeniowej procesora komputera, na którym zostanie uruchomiony. 
        
    \section{Zastosowane Algorytmy}
        \subsection{Opis teoretyczny}
            \subsubsection{Algorytm Sekwencyjny}
                
        \subsection{Realizacja praktyczna}
\end{document}